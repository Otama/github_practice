% Created 2016-08-16 火 10:53
\documentclass[a4paper,twoside,twocolumn]{bxjsarticle}
\usepackage{zxjatype}
\usepackage[ipa]{zxjafont}
\usepackage{xltxtra}
\usepackage{amsmath}
\usepackage{newtxtext,newtxmath}
\usepackage{graphicx}
\usepackage{hyperref}
\ifdefined\kanjiskip
\usepackage{pxjahyper}
\hypersetup{colorlinks=true}
\else
\ifdefined\XeTeXversion
\hypersetup{colorlinks=true}
\else
\ifdefined\directlua
\hypersetup{pdfencoding=auto,colorlinks=true}
\else
\hypersetup{unicode,colorlinks=true}
\fi
\fi
\fi

\usepackage{minted}
\usepackage[normalem]{ulem}
\author{産業技術大学院大学\\ 中鉢 欣秀}
\date{2016-08-14}
\title{GitHub入門(インストラクション)}
\hypersetup{
  pdfkeywords={},
  pdfsubject={},
  pdfcreator={Emacs 24.5.1 (Org mode 8.2.10)}}
\begin{document}

\maketitle

\section{GitHub入門インストラクション}
\label{sec-1}
\subsection{この授業について}
\label{sec-1-1}
これはGitの初心者が,基礎的なGitコマンドの利用方法から,
GitHubフローに基づく協同開発の方法までを学ぶ演習である.

事前に gitコマンドが利用できる環境を用意しておくこと.
またCUI端末でのshellによる基本的な操作を知っていると
スムーズに演習ができる.

第1章はGit初心者(初めてさわる者)を対象に基礎を学ぶ.
第2章は個人によるGitHubの初歩的な使い方を取り扱う.
第3章ではチームによるGitHubの使い方を知ろう.

\subsection{授業の進め方}
\label{sec-1-2}
\begin{itemize}
\item 演習の解説
\begin{itemize}
\item このインストラクションに従い,授業の進め方を説明する
\end{itemize}
\item Git/GitHubを学ぶ個人演習
\begin{itemize}
\item テキストの指示に従い,Git/Githubの使い方を学ぶ
\item テキストを完了したらTAに伝えること
\item その後,グループ演習に進む
\end{itemize}
\item 
\end{itemize}

\subsection{テキスト}
\label{sec-1-3}
\begin{itemize}
\item \href{./github_practice-handout.org}{Webページ}
\item \href{./github_practice-handout.pdf}{ハンドアウト(PDF)}
\end{itemize}

\subsection{提出物}
\label{sec-1-4}
% Emacs 24.5.1 (Org mode 8.2.10)
\end{document}