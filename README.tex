% Created 2016-08-16 火 14:23
\documentclass[a4paper,twoside,twocolumn]{bxjsarticle}
\usepackage{zxjatype}
\usepackage[ipa]{zxjafont}
\usepackage{xltxtra}
\usepackage{amsmath}
\usepackage{newtxtext,newtxmath}
\usepackage{graphicx}
\usepackage{hyperref}
\ifdefined\kanjiskip
\usepackage{pxjahyper}
\hypersetup{colorlinks=true}
\else
\ifdefined\XeTeXversion
\hypersetup{colorlinks=true}
\else
\ifdefined\directlua
\hypersetup{pdfencoding=auto,colorlinks=true}
\else
\hypersetup{unicode,colorlinks=true}
\fi
\fi
\fi

\usepackage{minted}
\usepackage[normalem]{ulem}
\author{産業技術大学院大学\\ 中鉢 欣秀}
\date{2016-08-18}
\title{GitHub入門(インストラクション)}
\hypersetup{
  pdfkeywords={},
  pdfsubject={},
  pdfcreator={Emacs 24.5.1 (Org mode 8.2.10)}}
\begin{document}

\maketitle

\section{GitHub入門}
\label{sec-1}
\subsection{この資料の入手先}
\label{sec-1-1}
\begin{itemize}
\item \url{https://github.com/ychubachi/github_practice}
\end{itemize}

\subsection{この授業について}
\label{sec-1-2}
\begin{itemize}
\item この授業ではGitの初心者が,基礎的なGitコマンドの利用方法から,
GitHub Flowに基づく協同開発の方法までを学ぶ
\end{itemize}

\subsection{事前準備}
\label{sec-1-3}
\begin{itemize}
\item 事前に gitコマンドが利用できる環境を用意しておくこと
\item CUI端末でのshellによる基本的な操作を知っているとスムーズに演習ができる
\end{itemize}

\subsection{授業の構成}
\label{sec-1-4}
\begin{itemize}
\item 個人演習では,テキストの指示に従い,
Git/GitHubを利用するにあたり必要となる知識を学ぶ
\item チーム演習では,GitHubを活用した協同開発の方法を深く学ぼう
\end{itemize}

\subsection{授業の進め方}
\label{sec-1-5}
\begin{enumerate}
\item 演習の解説
\begin{itemize}
\item 講師が授業の進め方を説明する
\end{itemize}
\item Git/GitHubを学ぶ個人演習
\begin{itemize}
\item 個人演習を通してGit/GitHubの使い方を学ぶ
\end{itemize}
\item チーム演習
\begin{itemize}
\item チームでの開発演習を実施する
\end{itemize}
\end{enumerate}

\section{個人演習}
\label{sec-2}
\subsection{個人演習のテキスト}
\label{sec-2-1}
\begin{itemize}
\item 個人演習のテキストは次のリンクから入手
\begin{itemize}
\item \href{https://github.com/ychubachi/github_practice/blob/master/github_practice-person_handout.org}{Webページ}
\item \href{https://github.com/ychubachi/github_practice/raw/master/github_practice-person_handout.pdf}{ハンドアウト(PDF)}
\end{itemize}
\end{itemize}

\section{個人演習からチーム演習へ}
\label{sec-3}
\subsection{個人演習からチーム演習への流れ}
\label{sec-3-1}
\begin{itemize}
\item この授業では最初に個人演習を行い,その後,チームによる演習に進む
\item その際,チーム編成が既に済んでいるか,または,
そうでないかで演習の進め方が異なる
\end{itemize}

\subsection{チーム編成が済んでいる場合}
\label{sec-3-2}
\begin{itemize}
\item 個人演習としてテキストの課題に取り組む
\item テキストを終えたメンバーは他のメンバーを積極的に助ける
\item 全員がテキストを終えることを目指す
\item 全員が完了,もしくは,時間になったらチーム演習に進む
\end{itemize}

\subsection{チームがまだできていない場合}
\label{sec-3-3}
\begin{itemize}
\item 後ほど席の移動をするので荷物をまとめておく
\item 個人演習としてテキストの課題に取り組む
\item テキストを完了したら講師・TAに伝えること
\item その後,チーム編成を経てチーム演習に進む
\end{itemize}

\subsection{チームがまだできていない場合の編成方法}
\label{sec-3-4}
\begin{itemize}
\item 個人演習が完了した者から順番に2名づつのペアを組んでいく
\item できたペアは空いている席に移動して,チーム演習を開始する
\item 受講者の半数がペアになったら,それ以降にテキストを終えた者は
既存のペアに追加していく
\item 最終的に3〜4人のグループにする
\end{itemize}

\section{チーム演習}
\label{sec-4}
\subsection{チーム演習のテキスト}
\label{sec-4-1}
\begin{itemize}
\item チーム演習のテキストは次のリンクから入手
\begin{itemize}
\item \href{https://github.com/ychubachi/github_practice/blob/master/github_practice-team_handout.org}{Webページ}
\item \href{https://github.com/ychubachi/github_practice/raw/master/github_practice-team_handout.pdf}{ハンドアウト(PDF)}
\end{itemize}
\end{itemize}

\section{成績評価の方法}
\label{sec-5}
\subsection{提出物}
\label{sec-5-1}
\begin{itemize}
\item 提出物は次のとおり
\begin{itemize}
\item 名前
\item 学籍番号
\item GitHubのアカウント名
\item GitHubのリポジトリのWeb URL
\begin{itemize}
\item 個人演習「our\_enpit」
\item チーム演習「team\_enpit」
\end{itemize}
\item 各自が行った作業の内容
\item 自己評価
\begin{itemize}
\item 5段階:5はとても優れている,4は優れている,3は普通,2は劣っている
\end{itemize}
\item 自己評価の理由
\item 演習全体の感想
\end{itemize}
\end{itemize}

\subsection{チーム演習の評価}
\label{sec-5-2}
\begin{itemize}
\item GitHubのリポジトリを対象に,主に以下の項目について評価する
\begin{itemize}
\item コミットの数(一人5以上)
\item コミットの粒度(意味のある単位でできるだけ細かく)
\item コミットメッセージの分かりやすさ
\item プルリクエストの活用
\item WikiやIssueの活用
\item コンフリクトの解消ができたか
\end{itemize}
\end{itemize}

\subsection{成果物の提出方法と補足資料}
\label{sec-5-3}
\begin{itemize}
\item 成果物の提出方法や,その他の補足資料はWikiを参照
\begin{itemize}
\item \href{https://github.com/ychubachi/github_practice/wiki}{Home · ychubachi/github\_practice Wiki}
\end{itemize}
\end{itemize}

\section{問い合わせ}
\label{sec-6}
\begin{itemize}
\item この授業の内容や資料に関する質問や問い合わせ,改善提案はGitHubのIssueに登録してください
\begin{itemize}
\item \href{https://github.com/ychubachi/github_practice/issues}{Issues · ychubachi/github\_practice}
\end{itemize}
\end{itemize}
% Emacs 24.5.1 (Org mode 8.2.10)
\end{document}