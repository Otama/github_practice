% Created 2016-08-16 火 10:12
\documentclass[a4paper,twoside,twocolumn]{bxjsarticle}
\usepackage{zxjatype}
\usepackage[ipa]{zxjafont}
\usepackage{xltxtra}
\usepackage{amsmath}
\usepackage{newtxtext,newtxmath}
\usepackage{graphicx}
\usepackage{hyperref}
\ifdefined\kanjiskip
\usepackage{pxjahyper}
\hypersetup{colorlinks=true}
\else
\ifdefined\XeTeXversion
\hypersetup{colorlinks=true}
\else
\ifdefined\directlua
\hypersetup{pdfencoding=auto,colorlinks=true}
\else
\hypersetup{unicode,colorlinks=true}
\fi
\fi
\fi

\usepackage{minted}
\usepackage[normalem]{ulem}
\author{産業技術大学院大学\\ 中鉢 欣秀}
\date{2016-08-14}
\title{GitHub入門}
\hypersetup{
  pdfkeywords={},
  pdfsubject={},
  pdfcreator={Emacs 24.5.1 (Org mode 8.2.10)}}
\begin{document}

\maketitle
\begin{abstract}
これはGitの初心者が,基礎的なGitコマンドの利用方法から,
GitHubフローに基づく協同開発の方法までを学ぶ演習である.

事前に gitコマンドが利用できる環境を用意しておくこと.
またCUI端末でのshellによる基本的な操作を知っていると
スムーズに演習ができる.

第1章はGit初心者(初めてさわる者)を対象に基礎を学ぶ.
第2章は個人によるGitHubの初歩的な使い方を取り扱う.
第3章ではチームによるGitHubの使い方を知ろう.
\end{abstract}

\section{Git入門}
\label{sec-1}
\subsection{Gitの操作方法と初期設定}
\label{sec-1-1}
\subsubsection{はじめに:Gitチートシート(カンニング表)}
\label{sec-1-1-1}
\begin{itemize}
\item 主なGitコマンドの一覧表
\begin{itemize}
\item \href{https://services.github.com/kit/downloads/ja/github-git-cheat-sheet.pdf}{Gitチートシート(日本語版)}
\end{itemize}
\item 必要に応じて印刷しておくとよい
\end{itemize}

\subsubsection{Gitコマンドの実行確認}
\label{sec-1-1-2}
\begin{itemize}
\item 端末を操作してGitコマンドを起動してみよう.
\item 次のとおり操作することでGitのバージョン番号が確認できる.
\end{itemize}

\begin{minted}[frame=single,linenos=true]{bash}
git --version
\end{minted}

\subsubsection{名前とメールアドレスの登録}
\label{sec-1-1-3}
\begin{itemize}
\item 名前とメールアドレスを登録しておく
\item 次のコマンドの\$NAMEと\$EMAILを各自の名前とメールアドレスに置き換えて実行せよ
\begin{itemize}
\item 名前はローマ字で設定すること
\end{itemize}
\end{itemize}

\begin{minted}[frame=single,linenos=true]{bash}
git config --global user.name $NAME
git config --global user.email $EMAIL
\end{minted}

\subsubsection{その他の設定}
\label{sec-1-1-4}
\begin{itemize}
\item 次のとおり,設定を行っておく
\begin{itemize}
\item 1行目:色付きで表示を見やすく
\item 2行目:pushする方法(詳細省略)
\end{itemize}
\end{itemize}

\begin{minted}[frame=single,linenos=true]{bash}
git config --global color.ui auto
git config --global push.default simple
\end{minted}

\subsubsection{設定の確認方法}
\label{sec-1-1-5}
\begin{itemize}
\item ここまでの設定を確認する
\end{itemize}

\begin{minted}[frame=single,linenos=true]{bash}
git config -l
\end{minted}

\subsection{Gitのリポジトリ}
\label{sec-1-2}
\subsubsection{プロジェクト用のディレクトリ}
\label{sec-1-2-1}
\begin{itemize}
\item リポジトリとはプロジェクトでソースコードなどを
配置するディレクトリ
\item Gitのリポジトリバージョン管理ができるようになる
\item GitHubと連携させることで共同作業ができる
\end{itemize}

\subsubsection{Gitリポジトリを利用するには}
\label{sec-1-2-2}
\begin{itemize}
\item リポジトリを利用する方法には主に2種類ある
\begin{enumerate}
\item git initコマンドで初期化する方法
\item git cloneコマンドでGitHubから入手する方法
\end{enumerate}
\item 本章では1.について解説する(次章からは2.で行う)
\end{itemize}

\subsubsection{Gitリポジトリの初期化方法}
\label{sec-1-2-3}
\begin{itemize}
\item my\_projectディレクトリを作成し,
  Gitリポジトリとして初期化するコマンドは次のとおり
\begin{itemize}
\item 1〜2行目:ディレクトリを作成して移動
\item 3行目:ディレクトリをリポジトリとして初期化
\end{itemize}
\end{itemize}

\begin{minted}[frame=single,linenos=true]{bash}
mkdir ~/my_project
cd ~/my_project
git init
\end{minted}

\begin{itemize}
\item 以降の作業は作成したmy\_projectディレクトリで行うこと
\begin{itemize}
\item 現在のディレクトリは「pwd」コマンドで確認できる
\end{itemize}
\end{itemize}

\subsubsection{リポジトリの状態を確認する方法}
\label{sec-1-2-4}
\begin{itemize}
\item 現在のリポジトリの状態を確認するコマンドは次のとおり
\end{itemize}

\begin{minted}[frame=single,linenos=true]{bash}
git status
\end{minted}

\begin{itemize}
\item このコマンドは頻繁に使用する
\item 何かうまく行かないことがあったら,このコマンドで状態を確認する癖を
つけるとよい
\begin{itemize}
\item 表示される内容の意味は徐々に覚えていけば良い
\end{itemize}
\end{itemize}

\subsubsection{「.git」ディレクトリを壊すべからず}
\label{sec-1-2-5}
\begin{itemize}
\item ティレクトリにリポジトリを作成すると「.git」という隠しディレクトリが
できる
\begin{itemize}
\item ls -aで確認できるが・・・
\end{itemize}
\item このディレクトリは絶対に, \uline{手動で変更してはならない}
\begin{itemize}
\item むろん,削除もしてはならない
\end{itemize}
\end{itemize}

\subsection{コミットの作成方法}
\label{sec-1-3}
\subsubsection{コミットについて}
\label{sec-1-3-1}
\begin{itemize}
\item Gitの用語における「コミット」とは,「ひとかたまりの作業」をいう
\begin{itemize}
\item 新しい機能を追加した,バグを直した,ドキュメントの内容を更新した,など
\end{itemize}
\item Gitは作業の履歴を,コミットを単位として管理する
\begin{itemize}
\item コミットは次々にリポジトリに追加されていき,これらを記録することで
バーションの管理ができる(古いバージョンに戻る,など)
\end{itemize}
\item コミットには,作業の内容を説明するメッセージをつける
\begin{itemize}
\item 更に,コミットには自動的にIDが振られることも覚えておくと良い
\end{itemize}
\end{itemize}

\subsubsection{READMEファイルの作成}
\label{sec-1-3-2}
\begin{itemize}
\item my\_projectリポジトリにREADMEファイルを作成してみよう
\end{itemize}

\begin{minted}[frame=single,linenos=true]{bash}
echo "My README file." > README
\end{minted}

\begin{itemize}
\item プロジェクトには \uline{必ずREADMEファイルを用意} しておくこと
\end{itemize}

\subsubsection{リポジトリの状態の確認}
\label{sec-1-3-3}
\begin{itemize}
\item git statusで現在のリポジトリの状態を確認する
\end{itemize}

\begin{minted}[frame=single,linenos=true]{bash}
git status
\end{minted}

\begin{itemize}
\item 未追跡のファイル(Untracked files:)の欄に作成したREADMEファイルが
(赤色で)表示される
\end{itemize}

\subsubsection{変更内容のステージング}
\label{sec-1-3-4}
\begin{itemize}
\item コミットの一つ手前にステージングという段階がある
\begin{itemize}
\item 変更をコミットするためには,ステージングしなくてはならない
\item 新しいファイルをステージングすると,これ以降,
gitがそのファイルの変更を追跡する
\end{itemize}
\end{itemize}

\subsubsection{ステージングの実行}
\label{sec-1-3-5}
\begin{itemize}
\item 作成したREADMEファイルをステージングするには,次のコマンドを打つ
\end{itemize}

\begin{minted}[frame=single,linenos=true]{bash}
git add .
\end{minted}

\begin{itemize}
\item 「git add」の「.(ピリオド)」を忘れないように
\begin{itemize}
\item ピリオドは,リポジトリにおけるすべての変更を意味する
\item 複数のファイルを変更した場合には,ファイル名を指定して
部分的にステージングすることもできる・・・
\begin{itemize}
\item が,このやりかたは好ましくない
\item 一度に複数の変更を行うのではなく,一つの変更を終えたら
こまめにコミットする
\end{itemize}
\end{itemize}
\end{itemize}

\subsubsection{ステージング後のリポジトリへの状態}
\label{sec-1-3-6}
\begin{itemize}
\item 再度,git statusコマンドで状態を確認しよう
\end{itemize}

\begin{minted}[frame=single,linenos=true]{bash}
git status
\end{minted}

\begin{itemize}
\item コミットされる変更(Changes to be committed:)の欄に,READMEファイルが
(緑色で)表示されれば正しい結果である
\end{itemize}

\subsubsection{ステージングされた内容をコミットする}
\label{sec-1-3-7}
\begin{itemize}
\item ステージング段階にある変更内容をコミットする
\item コミットにはその内容を示すメッセージ文をつける
\item 「First commit」というメッセージをつけて新しいコミットを作成する
\begin{itemize}
\item 「-m」オプションはそれに続く文字列をメッセージとして付与することを
指示するもの
\end{itemize}
\end{itemize}

\begin{minted}[frame=single,linenos=true]{bash}
git commit -m 'First commit'
\end{minted}

\subsubsection{コミット後の状態の確認}
\label{sec-1-3-8}
\begin{itemize}
\item コミットが正常に行われたことを確認する
\begin{itemize}
\item ここでもgit statusコマンドか活躍する
\end{itemize}
\end{itemize}

\begin{minted}[frame=single,linenos=true]{bash}
git status
\end{minted}

\begin{itemize}
\item 「nothing to commit, \ldots{}」との表示から
コミットすべきものがない(=過去の変更はコミットされた)ことが
わかる
\item この表示がでたら(無事コミットできたので)一安心してよい
\end{itemize}

\subsection{変更履歴の作成}
\label{sec-1-4}
\subsubsection{更なるコミットを作成する}
\label{sec-1-4-1}
\begin{itemize}
\item リポジトリで変更作業を行い,新しいコミットを追加する
\begin{itemize}
\item READMEファイルに新しい行を追加する
\end{itemize}
\item 次の\$NAMEをあなたの名前に変更して実行しなさい
\end{itemize}

\begin{minted}[frame=single,linenos=true]{bash}
echo $NAME >> README
\end{minted}

\begin{itemize}
\item 既存のファイルへの追加なので「>>」を用いていることに注意
\end{itemize}

\subsubsection{変更後の状態の確認}
\label{sec-1-4-2}
\begin{itemize}
\item リポジトリの状態をここでも確認する
\end{itemize}

\begin{minted}[frame=single,linenos=true]{bash}
git status
\end{minted}

\begin{itemize}
\item コミットのためにステージされていない変更(Changes not staged for commit:)の
欄に,変更された(modified)ファイルとしてREADMEが表示される
\end{itemize}

\subsubsection{差分の確認}
\label{sec-1-4-3}
\begin{itemize}
\item トラックされているファイルの変更箇所を確認する
\end{itemize}

\begin{minted}[frame=single,linenos=true]{bash}
git diff
\end{minted}

\begin{itemize}
\item 頭に「+」のある(緑色で表示された)行が新たに追加された内容を示す
\begin{itemize}
\item 削除した場合は「-」がつく
\end{itemize}
\end{itemize}

\subsubsection{新たな差分をステージングする}
\label{sec-1-4-4}
\begin{itemize}
\item 作成した差分をコミットできるようにするために,ステージング段階に上げる
\end{itemize}

\begin{minted}[frame=single,linenos=true]{bash}
git add .
\end{minted}

\begin{itemize}
\item git statusを行い,READMEファイルが「Changed to be commited:」の欄に
(緑色で)表示されていることを確認する
\item ステージさせるとgit diffの結果が空になる
\begin{itemize}
\item この場合,「git diff --staged」で確認可能
\end{itemize}
\end{itemize}

\subsubsection{ステージングされた新しい差分のコミット}
\label{sec-1-4-5}
\begin{itemize}
\item 変更内容を示すメッセージとともにコミットする
\end{itemize}

\begin{minted}[frame=single,linenos=true]{bash}
git commit -m 'Add my name'
\end{minted}

\subsection{履歴の確認}
\label{sec-1-5}
\subsubsection{バージョン履歴の確認}
\label{sec-1-5-1}
\begin{itemize}
\item これまでの変更作業の履歴を確認
\begin{itemize}
\item 2つのコミットが存在する
\end{itemize}
\end{itemize}

\begin{minted}[frame=single,linenos=true]{bash}
git log
\end{minted}

\begin{itemize}
\item 各コミットごとに表示される内容
\begin{itemize}
\item コミットのID(commit に続く英文字と数字の列)
\item AuthorとDate
\item コミットメッセージ
\end{itemize}
\end{itemize}

\subsubsection{一つのファイルの履歴}
\label{sec-1-5-2}
\begin{itemize}
\item 将来,複数のファイルを履歴管理するようになったら特定のファイルの
履歴のみ確認したい
\item その場合,次のとおりにする
\end{itemize}

\begin{minted}[frame=single,linenos=true]{bash}
git log --follow README
\end{minted}

\subsubsection{2つのコミットの比較}
\label{sec-1-5-3}
\begin{itemize}
\item 異なる2つのコミットの変更差分は次のコマンドで確認できる
\begin{itemize}
\item コミットのIDはlogで確認できる(概ね先頭4文字でよい)
\item ブランチごとの比較もできる(後述)
\end{itemize}
\end{itemize}

\begin{minted}[frame=single,linenos=true]{bash}
git diff $COMMIT_ID_1 $COMMIT_ID_2
\end{minted}

\subsubsection{コミットの情報確認}
\label{sec-1-5-4}
\begin{itemize}
\item 次のコマンドでコミットで行った変更内容が確認できる
\end{itemize}

\begin{minted}[frame=single,linenos=true]{bash}
git show $COMMIT_ID
\end{minted}

\subsection{ブランチの使い方}
\label{sec-1-6}
\subsubsection{ブランチとは}
\label{sec-1-6-1}
\begin{itemize}
\item 「ひとまとまりの作業」を行う場所
\item ソースコードなどの編集作業を始める際には
必ず新しいブランチを作成する
\end{itemize}

\subsubsection{masterは大事なブランチ}
\label{sec-1-6-2}
\begin{itemize}
\item Gitリポジトリの初期化後,最初のコミットを行うとmasterブランチができる
\item 非常に重要なブランチであり,
ここで \uline{直接編集作業を行ってはならない}
\begin{itemize}
\item ただし,本演習や,個人でGitを利用する場合はこの限りではない
\end{itemize}
\end{itemize}

\subsubsection{ブランチの作成と移動}
\label{sec-1-6-3}
\begin{itemize}
\item 新しいブランチ「new\_branch」を作成して,なおかつ,そのブランチに移動する
\begin{itemize}
\item 「-b」オプションで新規作成
\item オプションがなければ単なる移動(後述)
\end{itemize}
\end{itemize}

\begin{minted}[frame=single,linenos=true]{bash}
git checkout -b new_branch
\end{minted}

\begin{itemize}
\item 本来,ブランチには「これから行う作業の内容」が分かるような名前を付ける
\end{itemize}

\subsubsection{ブランチの確認}
\label{sec-1-6-4}
\begin{itemize}
\item ブランチの一覧と現在のブランチを確認する
\begin{itemize}
\item もともとあるmasterと,新しく作成したnew\_branchが表示される
\end{itemize}
\end{itemize}

\begin{minted}[frame=single,linenos=true]{bash}
git branch -vv
\end{minted}

\begin{itemize}
\item ブランチに紐づくコミットのIDが同じことも確認
\item git statusの一行目にも現在のブランチが表示される
\end{itemize}

\subsubsection{ブランチでのコミット作成}
\label{sec-1-6-5}
\begin{itemize}
\item READMEに現在の日時を追加
\end{itemize}

\begin{minted}[frame=single,linenos=true]{bash}
date >> README
git add .
git commit -m 'Add date'
\end{minted}

\begin{itemize}
\item 新しいコミットが追加できたことをgit logで確認
\item git branch -vvでコミットのIDが変化したことも確認
\end{itemize}

\subsubsection{ブランチの移動}
\label{sec-1-6-6}
\begin{itemize}
\item new\_branchブランチでコミットした内容をmasterに反映させる
\begin{itemize}
\item まずはmasterに移動する
\end{itemize}
\end{itemize}

\begin{minted}[frame=single,linenos=true]{bash}
git checkout master
\end{minted}

\begin{itemize}
\item git status,git branch -vvで現在のブランチを確認すること
\item この段階では,READMEファイルに行った変更が \uline{反映されてない} ことを
確認すること
\end{itemize}

\subsubsection{変更をmasterにマージ}
\label{sec-1-6-7}
\begin{itemize}
\item new\_branchで行ったコミットをmasterに反映させる
\end{itemize}

\begin{minted}[frame=single,linenos=true]{bash}
git merge new_branch
\end{minted}

\begin{itemize}
\item READMEに更新が反映されたことを確認
\item git branch -vvにより両ブランチのコミットIDが同じになったことも確認
\item git logも確認しておきたい
\end{itemize}

\subsubsection{マージ済みブランチの削除}
\label{sec-1-6-8}

\begin{itemize}
\item マージしたブランチはもはや不要なので削除して良い
\end{itemize}

\begin{minted}[frame=single,linenos=true]{bash}
git branch -d new_branch
\end{minted}

\begin{itemize}
\item git branch -vvコマンドで削除を確認
\end{itemize}

\subsection{{\bfseries\sffamily TODO} 操作を取り消すコマンド}
\label{sec-1-7}
\subsubsection{ステージング/コミットの修正}
\label{sec-1-7-1}
ファイルのステージングを取り消す

\begin{minted}[frame=single,linenos=true]{bash}
git reset $FILE
\end{minted}

\$COMMIT\_IDより後のコミットの取り消し(ローカルは保存)

\begin{minted}[frame=single,linenos=true]{bash}
git reset $COMMIT_ID
\end{minted}

\$COMMIT\_IDより後のコミットの取り消し(ローカルの変更も破棄)

\begin{minted}[frame=single,linenos=true]{bash}
git reset --hard $COMMIT_ID
\end{minted}

\section{GitHub入門}
\label{sec-2}
\subsection{GitHubとは}
\label{sec-2-1}
\subsubsection{GitHubでソーシャルコーディング}
\label{sec-2-1-1}
\begin{itemize}
\item ソーシャルコーディングのためのクラウド環境
\begin{itemize}
\item \href{https://github.com/}{GitHub}
\item \href{http://github.co.jp/}{GitHub Japan}
\end{itemize}
\item GitHubが提供する主な機能
\begin{itemize}
\item GitHub flowによる協同開発
\item Pull requests
\item Issue / Wiki
\end{itemize}
\end{itemize}

\subsubsection{GitHubアカウントの作成}
\label{sec-2-1-2}
\begin{itemize}
\item まず,次のURLの指示に従いアカウントを作成
\begin{itemize}
\item \href{https://help.github.com/articles/signing-up-for-a-new-github-account/}{Signing up for a new GitHub account - User Documentation}
\end{itemize}
\item アカウントの種類
\begin{itemize}
\item 無料版で作成する場合「Join GitHub for Free」を選択する
\item 学生の場合「Student Developer Pack」にアップグレードすることもできる
\end{itemize}
\item その後,確認メールが届くので,必要に応じて残りの手順を実施せよ
\begin{itemize}
\item \href{https://help.github.com/categories/setup/}{GitHub Help}
\end{itemize}
\end{itemize}

\subsubsection{SSHによるGitHubアクセス}
\label{sec-2-1-3}
\begin{itemize}
\item GitHubへのアクセスはSSHを用いた公開鍵暗号方式の認証を用いる
\begin{itemize}
\item SSH公開鍵の設定を行えば以降のパスワード認証が不要になる
\end{itemize}
\item SSHを生成してGitHubに登録しなさい
\begin{itemize}
\item 鍵を生成するとき「passphrases」が聞かれるが,この演習では何も入力しなくてよい
\item \href{https://help.github.com/articles/generating-an-ssh-key/}{Generating an SSH key - User Documentation}
\end{itemize}
\item もしSSHの登録がうまく行かなかったら,HTTPSを用いて接続し,GitHubのユーザ名と
パスワードでアクセスできる
\end{itemize}

\subsection{リモートリポジトリ}
\label{sec-2-2}
\subsubsection{リモート VS ローカルリポジトリ}
\label{sec-2-2-1}
\begin{itemize}
\item ローカルリポジトリ
\begin{itemize}
\item git initコマンドを用いて作成したリポジトリを「ローカルリポジトリ」という
\end{itemize}
\item リモートリポジトリ
\begin{itemize}
\item 「リモートリポジトリ」とは,サーバ上にあるリポジトリであり,
ローカルのリポジトリと連携させることができる
\end{itemize}
\item リモートリポジトリの利点
\begin{itemize}
\item ネットワークを経由してどこからでも利用することができる
\item 複数人のチームで協同作業をするときに活用できる
\end{itemize}
\end{itemize}
\subsubsection{リモートリポジトリの作成}
\label{sec-2-2-2}
\begin{itemize}
\item リモートリポジトリをGitHubで作成する
\item 名前は「our\_project」とする
\item 次の手順で作成する
\begin{itemize}
\item \href{https://help.github.com/articles/creating-a-new-repository/}{Creating a new repository - User Documentation}
\end{itemize}
\end{itemize}

\subsection{GitHub flow}
\label{sec-2-3}
\subsubsection{GitHub flowwによる開発の流れ}
\label{sec-2-3-1}
\begin{itemize}
\item GitHub flow
\begin{itemize}
\item \href{https://guides.github.com/introduction/flow/}{Understanding the GitHub Flow · GitHub Guides}
\end{itemize}
\item 言葉による説明
\begin{enumerate}
\item リモートリポジトリをローカルに複製
\item masterから作業用ブランチを作成
\item ブランチで編集作業
\item ブランチでコミットの作成
\item ブランチをリモートに送る
\item GitHubでプルリクエストを作成
\item GitHubでレビュー(+自動テスト)
\item GitHubでプルリクエストをマージ
\item ローカルのmasterを最新にする(2.に戻る)
\end{enumerate}
\end{itemize}

\subsubsection{1: リモートリポジトリをローカルに複製}
\label{sec-2-3-2}
\begin{itemize}
\item リモートにあるリポジトリをローカルに複製することをcloneという
\begin{itemize}
\item \href{https://help.github.com/articles/cloning-a-repository/}{Cloning a repository - User Documentation}
\end{itemize}
\item 下記の「\$GITHUB\_URL」の部分をGitHubにあるour\_projectリポジトリのURLにして実行
\begin{itemize}
\item リモートのURLはブラウザで確認する
\item ssh接続の場合,URLは「git@\ldots{}」(HTTPSの場合「https:\ldots{}」
\end{itemize}
\end{itemize}

\begin{minted}[frame=single,linenos=true]{bash}
cd ~
git clone $GITHUB_URL
cd our_project
\end{minted}

\begin{itemize}
\item この作業は基本的にはプロジェクトに対して一度だけ行うこと
\begin{itemize}
\item 別なマシンで作業したいときなどは話は別
\end{itemize}
\end{itemize}

\subsubsection{2: masterから作業用ブランチを作成}
\label{sec-2-3-3}
\begin{itemize}
\item 作業用のブランチを作成して移動する
\begin{itemize}
\item ブランチの名前は「greeting」とする
\end{itemize}
\end{itemize}

\begin{minted}[frame=single,linenos=true]{bash}
git checkout -b greeting
\end{minted}

\subsubsection{3: ブランチで編集作業を行う}
\label{sec-2-3-4}
\begin{itemize}
\item ここでは,hello.txtという名前のファイルを作成する
\end{itemize}

\begin{minted}[frame=single,linenos=true]{bash}
echo 'Hello GitHub' > hello.txt
\end{minted}

\subsubsection{4: ブランチでコミットを作成}
\label{sec-2-3-5}
\begin{itemize}
\item 変更した内容をステージングしてからコミットする
\end{itemize}

\begin{minted}[frame=single,linenos=true]{bash}
git add .
git commit -m 'Create hello.txt'
\end{minted}

\begin{itemize}
\item この編集,add,commitの作業は作業が一区切りつくまで何回も繰り返してよい…
\begin{itemize}
\item が,こまめにpushするのが良いとされる
\end{itemize}
\end{itemize}

\subsubsection{5: ブランチをリモートに送る}
\label{sec-2-3-6}
\begin{itemize}
\item ブランチで作成したコミットをリモートに送る(push)
\begin{itemize}
\item 下記のoriginはリポジトリのURLの別名として自動で設定されているもの
\item greetingは作業しているブランチ名
\end{itemize}
\end{itemize}

\begin{minted}[frame=single,linenos=true]{bash}
git push -u origin greeting
\end{minted}

\subsubsection{6. GitHubでプルリクエストを送る}
\label{sec-2-3-7}
\begin{itemize}
\item ブランチがGitHubに登録されたことを確認し,Pull requestを作成する
\item 手順は次のとおり
\begin{itemize}
\item \href{https://help.github.com/articles/using-pull-requests/}{Using pull requests - User Documentation} の前半
\item \href{https://help.github.com/articles/creating-a-pull-request/}{Creating a pull request - User Documentation}
\end{itemize}
\end{itemize}

\subsubsection{7. GitHubでレビュー(+自動テスト)}
\label{sec-2-3-8}
\begin{itemize}
\item プルリクエストを用いたレビューの方法は下記参照
\begin{itemize}
\item \href{https://help.github.com/articles/using-pull-requests/}{Using pull requests - User Documentation} の後半
\end{itemize}
\item 人手によるレビューの他,自動的なテストも行うのが望ましい
\begin{itemize}
\item 説明は省略
\end{itemize}
\end{itemize}

\subsubsection{8. GitHubでプルリクエストをマージ}
\label{sec-2-3-9}
\begin{itemize}
\item Pull requestのレビューが済んだらマージする
\begin{itemize}
\item \href{https://help.github.com/articles/merging-a-pull-request/}{Merging a pull request - User Documentation}
\end{itemize}
\item マージが完了したら,ローカル・リモート共に,マージ済みのブランチは削除してよい
\end{itemize}

\subsubsection{9. ローカルのmaster を最新版にする}
\label{sec-2-3-10}

\begin{itemize}
\item GitHubで行ったマージをローカルに反映させる
\begin{itemize}
\item masterブランチに移動してgit pull
\item 不要になった作業用ブランチは削除
\end{itemize}
\end{itemize}

\begin{minted}[frame=single,linenos=true]{bash}
git checkout master
git pull
git branch -d greeting
\end{minted}

\begin{itemize}
\item 練習のため,ここで手順2:に戻り,一連の作業を複数回繰り返すこと
\begin{itemize}
\item \uline{体に叩き込む!}
\end{itemize}
\end{itemize}

\subsection{コンフリクトについて}
\label{sec-2-4}
\subsubsection{GitHub flow におけるコンフリクトについて}
\label{sec-2-4-1}
\begin{itemize}
\item コンフリクトとは?
\begin{itemize}
\item コンフリクトは、コードの同じ箇所を複数の人が別々に編集すると発生
\end{itemize}
\item コンフリクトが起きると?
\begin{itemize}
\item GitHub に提出した Pull requests が自動的にマージできない
\end{itemize}
\item 基本的な対処法
\begin{itemize}
\item 初心者は、演習の最初の方では「他人と同じファイルを編集しない」こと
にして、操作になれる
(上達したら積極的にコンフリクトを起こしてみて、その解決方法を学ぶ)
\item コミットはできるだけ細かく作成すると良い
(その分,他の人とかち合う可能性が減る)
\item Pull requests でコンフリクトが発生し、自動的にマージできない状態に
なったら、 その PR を送った人がコンフリクトを自分で解消する
(あるいは解消方法をメンバーに聞く)
\end{itemize}
\end{itemize}
\subsubsection{GitHubでのコンフリクトの解消方法}
\label{sec-2-4-2}
\begin{itemize}
\item new\_feature ブランチで作業中であり、最新の更新は commit 済とする

\item 解消するための操作は次のとおり
\begin{itemize}
\item 1:一度masterブランチに移動.2:手元のmasterを最新版に.3:作業中のブランチへ.
4:ここでmasterを手動でマージ.コンフリクトが発生するので解消する.
5:このブランチを再度push
\end{itemize}
\item これにより,プルリクエストがマージ可能になれば成功
\end{itemize}

\begin{minted}[frame=single,linenos=true]{bash}
git checkout master
git pull origin master
git checkout new_feature
git merge master
git push origin new_feature
\end{minted}

\subsubsection{コンフリクト解消の練習}
\label{sec-2-4-3}
TODO
\section{{\bfseries\sffamily TODO} GitHubによるチーム開発}
\label{sec-3}
\subsection{{\bfseries\sffamily TODO} チーム開発}
\label{sec-3-1}
\subsubsection{チーム編成}
\label{sec-3-1-1}
\begin{itemize}
\item ここまでの演習内容が終わったものは教員かTAに教えること
\item 終わったものから順番にチームを編成する
\item チームができたら代表者1名がGitHubでリポジトリを作成する
\begin{itemize}
\item 名前は「team\_project」とする
\end{itemize}
\end{itemize}

\subsubsection{コラボレーターの追加}
\label{sec-3-1-2}
\begin{itemize}
\item 代表者は残りのメンバーを協同作業者(コラボレータ)として追加する
\begin{itemize}
\item GitHubのリポジトリをブラウザで開く.
\item Settings -> Collaborators を選ぶ
\item メンバーを招待する
\item 招待されたメンバーには確認のメールが届くので,リンクをクリックする
\end{itemize}
\end{itemize}

\section{{\bfseries\sffamily TODO} 演習課題}
\label{sec-4}
\subsection{ペアで行う GitHub}
\label{sec-4-1}
\subsubsection{課題1:ペアで GitHub を使ってみよう}
\label{sec-4-1-1}
\begin{enumerate}
\item 隣同士でペアを組む
\item レポジトリを作成する(どちらか一方)
\begin{itemize}
\item \texttt{bundle gem} でひな形を作る(初心者は Gem でなくても良い)
\end{itemize}
\item レポジトリの Collaborators に登録する
\item レポジトリに対して、次のことを行う
\begin{itemize}
\item Pull requests を利用してみる
\item Issue を利用してみる
\item Wiki を利用してみる
\end{itemize}
\end{enumerate}

\subsubsection{課題1の続き}
\label{sec-4-1-2}
\begin{enumerate}
\item Pull request \& merge の作業を各自5回以上行う
\begin{itemize}
\item ディスカッションやコードレビューもやってみる
\end{itemize}
\item Issue を5個以上登録する
\begin{itemize}
\item Pull request による Issue の close なども試す
\end{itemize}
\item Wiki でページを作成する
\begin{itemize}
\item ページを5つ程度作成して、リンクも貼る
\end{itemize}
\end{enumerate}

\subsection{グループで行う GitHub}
\label{sec-4-2}
\subsubsection{課題:グループで GitHub (1)}
\label{sec-4-2-1}
\begin{enumerate}
\item ペアを2つ組み合わせて4人グループを作成する
\begin{itemize}
\item 課題1が終わったペアから順番にグループ編成
\end{itemize}
\item 作りたい Gem について相談して仕様を決める
\begin{itemize}
\item テーマはなんでも良い
\begin{itemize}
\item Web API を利用したコマンドラインツールなど
\end{itemize}
\item ある程度の役割分担も決めておく
\end{itemize}
\item レポジトリを作成する(代表者1名)
\begin{itemize}
\item コラボレーターを追加する
\end{itemize}
\item 今まで学んだ知識を活用して Gem を開発する
\end{enumerate}

\subsubsection{課題:グループで GitHub (2)}
\label{sec-4-2-2}
\begin{enumerate}
\item グルーブメンバーでGemを共同で作成する
\item GitHub Flow の実践
\item Travis CI によるテストの自動化
\item RubyGems.org への自動ディプロイ
\item その他、GitHub の各種機能の活用
\end{enumerate}

\section{{\bfseries\sffamily TODO} 演習の成果物の提出}
\label{sec-5}
\subsection{{\bfseries\sffamily TODO} アカウントの作成}
\label{sec-5-1}
\subsubsection{課題}
\label{sec-5-1-1}
\href{https://github.com/}{GitHub} にアカウントを作成せよ
\subsubsection{提出}
\label{sec-5-1-2}
TODO: Google form
% Emacs 24.5.1 (Org mode 8.2.10)
\end{document}