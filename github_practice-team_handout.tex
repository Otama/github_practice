% Created 2016-08-16 火 20:22
\documentclass[a4paper,twoside,twocolumn]{bxjsarticle}
\usepackage{zxjatype}
\usepackage[ipa]{zxjafont}
\usepackage{xltxtra}
\usepackage{amsmath}
\usepackage{newtxtext,newtxmath}
\usepackage{graphicx}
\usepackage{hyperref}
\ifdefined\kanjiskip
\usepackage{pxjahyper}
\hypersetup{colorlinks=true}
\else
\ifdefined\XeTeXversion
\hypersetup{colorlinks=true}
\else
\ifdefined\directlua
\hypersetup{pdfencoding=auto,colorlinks=true}
\else
\hypersetup{unicode,colorlinks=true}
\fi
\fi
\fi

\usepackage{minted}
\usepackage[normalem]{ulem}
\author{産業技術大学院大学\\ 中鉢 欣秀}
\date{2016-08-18}
\title{GitHub入門(チーム演習)}
\hypersetup{
  pdfkeywords={},
  pdfsubject={},
  pdfcreator={Emacs 24.5.1 (Org mode 8.2.10)}}
\begin{document}

\maketitle

\section{チーム編集の準備}
\label{sec-1}
\subsection{演習のための準備}
\label{sec-1-1}
\subsubsection{概要}
\label{sec-1-1-1}
\begin{itemize}
\item この演習では,最終的に全員でHTMLによるWebサイトを作ることを目指す
\item Webサーバは使わず,スタティックなサイトで構わない
\item 可能ならばCSSやJavaScriptを使っても良いが必須ではない
\end{itemize}

\subsubsection{リポジトリの作成}
\label{sec-1-1-2}
\begin{itemize}
\item チームができたら代表者1名がGitHubでリポジトリを作成する
\begin{itemize}
\item 名前は「team\_project」とする
\end{itemize}
\end{itemize}

\subsubsection{コラボレーターの追加}
\label{sec-1-1-3}
\begin{itemize}
\item 代表者は残りのメンバーを協同作業者(コラボレータ)として追加する
\begin{itemize}
\item \href{https://help.github.com/articles/inviting-collaborators-to-a-personal-repository/}{Inviting collaborators to a personal repository - User Documentation}
\end{itemize}
\item 招待されたメンバーには確認のメールが届く
\begin{itemize}
\item これにより,全員がGitHubのリポジトリにpushできるようになる
\end{itemize}
\end{itemize}

\subsubsection{リポジトリのclone}
\label{sec-1-1-4}
\begin{itemize}
\item 全員,リポジトリをローカルにcloneする
\begin{itemize}
\item \href{https://help.github.com/articles/cloning-a-repository/}{Cloning a repository - User Documentation}
\end{itemize}
\end{itemize}

\begin{minted}[frame=single,linenos=true]{bash}
cd ~
git clone $GITHUB_URL
cd team_project
\end{minted}

\section{チーム演習}
\label{sec-2}
\subsection{チーム演習について}
\label{sec-2-1}
\subsubsection{課題1: GitHubのIssue/Wikiを学ぶ}
\label{sec-2-1-1}
\begin{itemize}
\item リポジトリのIssue機能を使ってみよう
\begin{itemize}
\item 一人1つIssueを登録する
\item メンバーのIssueに挨拶する(投稿する)
\item 終わったらIssueを閉じてみる
\end{itemize}
\item リポジトリのWikiを使ってみよう
\begin{itemize}
\item Wikiを使ってチームメンバーの自己紹介をしてみよう
\end{itemize}
\item なお,この演習にあまり時間をかけてはならない
\end{itemize}

\subsubsection{課題2: まずは全員1回コミットしよう}
\label{sec-2-1-2}
[この課題では練習のためmasterブランチをそのまま使う]

\begin{itemize}
\item 全員,1つファイルをコミットしてプッシュする
\item まず,コミットするファイルを作る
\item ファイル名は各自のGitHubのアカウント名+「.html」とする
\item 下記の\$MY\_FILEをファイル名に置き換えて実行
\item コミットメッセージも書く
\end{itemize}

\begin{minted}[frame=single,linenos=true]{bash}
git add $MY_FILE
git commit -m '<メッセージ>'
git push
\end{minted}

\begin{itemize}
\item 全員完了したらローカルのmasterブランチを最新にする
\end{itemize}

\begin{minted}[frame=single,linenos=true]{bash}
git pull
\end{minted}

\begin{itemize}
\item ローカルにファイルができているか確認
\end{itemize}

\subsubsection{課題3: ブランチをpushする}
\label{sec-2-1-3}
\begin{itemize}
\item 全員1回,最初のGitHub Flowを成功させよう
\item まず,masterが最新版であることを確認
\item 「作業の内容がわかりやすい名前」でブランチを作る
\item \$MY\_FILEに,中身を追加してみよう(内容は何でも良い)
\item git add/commit/pushを正確に実行しよう
\item ブランチが無事pushできたら,GitHubをブラウザで確認する
\end{itemize}

\subsubsection{課題4: いよいよプルリク}
\label{sec-2-1-4}
\begin{itemize}
\item プルリクエストを出してみよう
\item 他のメンバーのプルリクエストにコメントしてみよう
\item コメントには顔文字なども利用できるので活用してみよう
\begin{itemize}
\item やり方は各自で調べること
\end{itemize}
\end{itemize}

\subsubsection{課題5: そしてマージ}
\label{sec-2-1-5}
\begin{itemize}
\item マージしてみよう
\item この段階でコンフリクトが出ることはないはず
(同じファイルを編集していない)だが,もし
マージできない場合は,プルリクエストを削除し,
課題3からやり直す
\end{itemize}

\subsubsection{課題6: 何回も繰り返す}
\label{sec-2-1-6}
\begin{itemize}
\item 同じファイルに更なる変更を加え,GitHub Flowを回してみよう
\item これを最低3回は繰り返したい
\end{itemize}

\subsubsection{課題7: ぼちぼちコンフリクト}
\label{sec-2-1-7}
\begin{itemize}
\item 誰かが空の「index.html」ファイルを作成する
\item 全員でindex.htmlを編集してみよう
\begin{itemize}
\item \$MY\_FILEへのリンクを貼る
\end{itemize}
\item pushしてプルリクエストを出してみる
\item 何人かはコンフリクトになるはずだ
\end{itemize}

\subsubsection{課題8: コンフリクトの解消}
\label{sec-2-1-8}
\begin{itemize}
\item コンフリクトが出たメンバーは,それを解消してみよう
\item コンフリクトが出なかったメンバーは,コンフリクトが出ているメンバーの
作業を見る
\begin{itemize}
\item 困っていたら助けてあげよう
\end{itemize}
\end{itemize}

\subsubsection{課題9: Webサイトを作ってみよう}
\label{sec-2-1-9}
\begin{itemize}
\item チームで内容を相談し,Webサイトを作ってみよう
\item index.htmlや\$MY\_FILE以外にもファイルを追加して
  素敵なWebサイトを作ろう
\end{itemize}

\subsubsection{注意事項}
\label{sec-2-1-10}
\begin{itemize}
\item 実は,GitHubでは,gitコマンドを使わなくても,
ブラウザベースでファイルのアップロードや編集,コミットの作成などが
できるが,このことに気がついてはならない
\begin{itemize}
\item 万が一,気がついてしまったものはしょうがないものとする
\end{itemize}
\end{itemize}
% Emacs 24.5.1 (Org mode 8.2.10)
\end{document}