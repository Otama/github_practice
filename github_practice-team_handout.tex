% Created 2016-08-16 火 14:23
\documentclass[a4paper,twoside,twocolumn]{bxjsarticle}
\usepackage{zxjatype}
\usepackage[ipa]{zxjafont}
\usepackage{xltxtra}
\usepackage{amsmath}
\usepackage{newtxtext,newtxmath}
\usepackage{graphicx}
\usepackage{hyperref}
\ifdefined\kanjiskip
\usepackage{pxjahyper}
\hypersetup{colorlinks=true}
\else
\ifdefined\XeTeXversion
\hypersetup{colorlinks=true}
\else
\ifdefined\directlua
\hypersetup{pdfencoding=auto,colorlinks=true}
\else
\hypersetup{unicode,colorlinks=true}
\fi
\fi
\fi

\usepackage{minted}
\usepackage[normalem]{ulem}
\author{産業技術大学院大学\\ 中鉢 欣秀}
\date{2016-08-18}
\title{GitHub入門(チーム演習)}
\hypersetup{
  pdfkeywords={},
  pdfsubject={},
  pdfcreator={Emacs 24.5.1 (Org mode 8.2.10)}}
\begin{document}

\maketitle

\section{{\bfseries\sffamily TODO} GitHubによるチーム開発}
\label{sec-1}
\subsection{{\bfseries\sffamily TODO} チーム開発}
\label{sec-1-1}
\subsubsection{チーム編成}
\label{sec-1-1-1}
\begin{itemize}
\item ここまでの演習内容が終わったものは教員かTAに教えること
\item 終わったものから順番にチームを編成する
\item チームができたら代表者1名がGitHubでリポジトリを作成する
\begin{itemize}
\item 名前は「team\_project」とする
\end{itemize}
\end{itemize}

\subsubsection{コラボレーターの追加}
\label{sec-1-1-2}
\begin{itemize}
\item 代表者は残りのメンバーを協同作業者(コラボレータ)として追加する
\begin{itemize}
\item GitHubのリポジトリをブラウザで開く.
\item Settings -> Collaborators を選ぶ
\item メンバーを招待する
\item 招待されたメンバーには確認のメールが届くので,リンクをクリックする
\end{itemize}
\end{itemize}

\section{{\bfseries\sffamily TODO} 演習課題}
\label{sec-2}
\subsection{ペアで行う GitHub}
\label{sec-2-1}
\subsubsection{課題1:ペアで GitHub を使ってみよう}
\label{sec-2-1-1}
\begin{enumerate}
\item 隣同士でペアを組む
\item レポジトリを作成する(どちらか一方)
\begin{itemize}
\item \texttt{bundle gem} でひな形を作る(初心者は Gem でなくても良い)
\end{itemize}
\item レポジトリの Collaborators に登録する
\item レポジトリに対して、次のことを行う
\begin{itemize}
\item Pull requests を利用してみる
\item Issue を利用してみる
\item Wiki を利用してみる
\end{itemize}
\end{enumerate}

\subsubsection{課題1の続き}
\label{sec-2-1-2}
\begin{enumerate}
\item Pull request \& merge の作業を各自5回以上行う
\begin{itemize}
\item ディスカッションやコードレビューもやってみる
\end{itemize}
\item Issue を5個以上登録する
\begin{itemize}
\item Pull request による Issue の close なども試す
\end{itemize}
\item Wiki でページを作成する
\begin{itemize}
\item ページを5つ程度作成して、リンクも貼る
\end{itemize}
\end{enumerate}

\subsection{グループで行う GitHub}
\label{sec-2-2}
\subsubsection{課題:グループで GitHub (1)}
\label{sec-2-2-1}
\begin{enumerate}
\item ペアを2つ組み合わせて4人グループを作成する
\begin{itemize}
\item 課題1が終わったペアから順番にグループ編成
\end{itemize}
\item 作りたい Gem について相談して仕様を決める
\begin{itemize}
\item テーマはなんでも良い
\begin{itemize}
\item Web API を利用したコマンドラインツールなど
\end{itemize}
\item ある程度の役割分担も決めておく
\end{itemize}
\item レポジトリを作成する(代表者1名)
\begin{itemize}
\item コラボレーターを追加する
\end{itemize}
\item 今まで学んだ知識を活用して Gem を開発する
\end{enumerate}

\subsubsection{課題:グループで GitHub (2)}
\label{sec-2-2-2}
\begin{enumerate}
\item グルーブメンバーでGemを共同で作成する
\item GitHub Flow の実践
\item Travis CI によるテストの自動化
\item RubyGems.org への自動ディプロイ
\item その他、GitHub の各種機能の活用
\end{enumerate}

\section{{\bfseries\sffamily TODO} 演習の成果物の提出}
\label{sec-3}
\subsection{{\bfseries\sffamily TODO} アカウントの作成}
\label{sec-3-1}
\subsubsection{課題}
\label{sec-3-1-1}
\href{https://github.com/}{GitHub} にアカウントを作成せよ
\subsubsection{提出}
\label{sec-3-1-2}
TODO: Google form

チーム演習

全員が自分のGitHubアカウント名.htmtlを作りコミット
何回か修正を繰り返す

コンフリクトの演習
だれかがindex.htmlを作成
各自がindex.htmlに自分のhtmlファイルへのリンクをはる
マージするとコンフリクトが発生するので
% Emacs 24.5.1 (Org mode 8.2.10)
\end{document}